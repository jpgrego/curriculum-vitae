%% start of file `template.tex'.
%% Copyright 2006-2013 Xavier Danaux (xdanaux@gmail.com).
%
% This work may be distributed and/or modified under the
% conditions of the LaTeX Project Public License version 1.3c,
% available at http://www.latex-project.org/lppl/.


\documentclass[11pt,a4paper,sans]{moderncv}        % possible options include font size ('10pt', '11pt' and '12pt'), paper size ('a4paper', 'letterpaper', 'a5paper', 'legalpaper', 'executivepaper' and 'landscape') and font family ('sans' and 'roman')

% modern themes
\moderncvstyle{banking}                            % style options are 'casual' (default), 'classic', 'oldstyle' and 'banking'
\moderncvcolor{blue}                                % color options 'blue' (default), 'orange', 'green', 'red', 'purple', 'grey' and 'black'
%\renewcommand{\familydefault}{\sfdefault}         % to set the default font; use '\sfdefault' for the default sans serif font, '\rmdefault' for the default roman one, or any tex font name
%\nopagenumbers{}                                  % uncomment to suppress automatic page numbering for CVs longer than one page

% character encoding
\usepackage[utf8]{inputenc}                       % if you are not using xelatex ou lualatex, replace by the encoding you are using
%\usepackage{CJKutf8}                              % if you need to use CJK to typeset your resume in Chinese, Japanese or Korean

% adjust the page margins
\usepackage[scale=0.75]{geometry}
%\setlength{\hintscolumnwidth}{3cm}                % if you want to change the width of the column with the dates
%\setlength{\makecvtitlenamewidth}{10cm}           % for the 'classic' style, if you want to force the width allocated to your name and avoid line breaks. be careful though, the length is normally calculated to avoid any overlap with your personal info; use this at your own typographical risks...

\usepackage{import}

% personal data
\name{João}{Grego}
\title{Curriculum Vitae}                               % optional, remove / comment the line if not wanted
\phone[mobile]{+351 915569263}                   % optional, remove / comment the line if not wanted
\email{jpgrego@protonmail.com}                               % optional, remove / comment the line if not wanted
\homepage{https://github.com/jpgrego}                         % optional, remove / comment the line if not wanted
%\extrainfo{additional information}                 % optional, remove / comment the line if not wanted
%\photo[64pt][0.4pt]{picture}                       % optional, remove / comment the line if not wanted; '64pt' is the height the picture must be resized to, 0.4pt is the thickness of the frame around it (put it to 0pt for no frame) and 'picture' is the name of the picture file
%\quote{Some quote}                                 % optional, remove / comment the line if not wanted

% to show numerical labels in the bibliography (default is to show no labels); only useful if you make citations in your resume
%\makeatletter
%\renewcommand*{\bibliographyitemlabel}{\@biblabel{\arabic{enumiv}}}
%\makeatother
%\renewcommand*{\bibliographyitemlabel}{[\arabic{enumiv}]}% CONSIDER REPLACING THE ABOVE BY THIS

% bibliography with mutiple entries
%\usepackage{multibib}
%\newcites{book,misc}{{Books},{Others}}
%----------------------------------------------------------------------------------
%            content
%----------------------------------------------------------------------------------
\begin{document}
%\begin{CJK*}{UTF8}{gbsn}                          % to typeset your resume in Chinese using CJK
%-----       resume       ---------------------------------------------------------
\makecvtitle

\small{Aluno finalista do mestrado integrado de Engenharia de Computadores e Telemática, presentemente a desenvolver a dissertação de mestrado. Possui paixão por desenvolvimento de software e gestão de infraestrutura informática, com fortes competências técnicas e sociais para trabalhar em equipa.}

\section{Experiência profissional}

\vspace{6pt}

\begin{itemize}

\item{\cventry{Julho 2011}{Técnico de Redes}{Meo XL Party}{Braga}{}{\vspace{3pt}Integrou uma equipa responsável pela montagem e manutenção da rede interna do evento, tendo que garantir serviço e qualidade do sinal durante o decorrer deste.}}

\vspace{6pt}

\item{\cventry{Junho 2015 -- Presente}{Engenheiro de Software}{BEEVERYCREATIVE}{Aveiro}{}{Integra actualmente a equipa responsável pelo desenvolvimento de \textit{software} e \textit{firmware}} das impressoras da BEEVERYCREATIVE, empresa fabricante de impressoras 3D. Efectuou a manutenção de projectos existentes, melhorando significativamente a sua estabilidade e usabilidade, e desenvolveu projectos novos, em colaboração com entidades externas, como a Agência Espacial Europeia.}

\end{itemize}

\section{Educação}

\vspace{5pt}

\subsection{Qualificações académicas}

\vspace{5pt}

\begin{itemize}

\item{\cventry{2008 -- }{Mestrado Integrado em Engenharia de Computadores e Telemática}{Universidade de Aveiro}{Aveiro}{}{}}

\end{itemize}

\vspace{2pt}

\subsection{Projectos}

\vspace{5pt}

\begin{itemize}

\item{\textbf{Dissertação de mestrado (em progresso): } \textit{'Integrated monitoring in Android Devices'}

\vspace{3pt}

\small{Encontra-se de momento na fase de concepção e elaboração de um sistema de monitorização para o sistema operativo de dispositivos móveis \textit{Android}. Este sistema consiste numa aplicação com o objectivo de analisar a actividade do dispositivo em que se encontra instalado para, posteriormente, tentar detectar eventos que se afastem demasiado daquilo que é estabelecido ser o comportamento normal do dispositivo, o \textit{status quo} de operação do dispositivo. Este trabalho é, assim, uma tentativa de apresentar uma alternativa na detecção de \textit{malware} em dispositivos móveis. Escreveu um \textit{paper} sobre este trabalho para a \textit{ConfTele}, conferência que decorreu em Setembro de 2015, organizado pelo Instituto de Telecomunicações, da Universidade de Aveiro.}}

\newpage

\item{\textbf{Projecto de 3º ano: } \textit{'Attendance control'}

\vspace{3pt}

\small{Este projecto tinha como objectivo desenvolver um método eficiente para registar automaticamente a presença de estudantes em aulas, exames e outras actividades académicas relacionadas. No final, o projecto foi apresentado e o seu grupo reuniu-se com os Serviços de Tecnologias de Informação e Comunicação da Universidade de Aveiro de maneira a implementar o sistema oficialmente.}}

\vspace{6pt}

\item{\textbf{Projecto da cadeira de Engenharia de Software: }\textit{'Cambada in the Web'}

\vspace{3pt}

\small{Integrou uma equipa no âmbito da cadeira de Engenharia de Software, em colaboração com o CAMBADA (Cooperative Autonomous Mobile roBots with Advanced Distributed Architecture), a famosa equipa de futebol robótico da Universidade de Aveiro, de maneira a desenvolver uma plataforma web para observação de estatísticas e observação e gravação de jogos a decorrer.}}

\vspace{6pt}

\item{\textbf{Projecto da cadeira de Engenharia de Serviços: }\textit{'eLock'}

\vspace{3pt}

\small{O eLock foi desenvolvido durante a frequência da cadeira de Engenharia de Serviços. Juntamente com dois colegas, desenvolveu um sistema de fechadura electrónica seguro, eficiente e customizável, que permitisse a um utilizador abrir a fechadura com o seu dispositivo móvel, permitir a outros utilizadores abrirem a fechadura, e até especificar o horário a que um utilizador podia abrir determinada fechadura.}}

\end{itemize}

\section{Outras experiências}

\vspace{6pt}

\begin{itemize}
\item{\cventry{Maio 2012}{Participante}{Micro-rato}{Universidade de Aveiro}{}{\vspace{3pt}Participou, juntamente com um colega, nesta competição, que consiste em programar um robot para encontrar o caminho que deve percorrer para chegar ao final de um labirinto.}}

\vspace{6pt}

\item{\cventry{2013 -- 2014}{Membro fundador}{Núcleo de Robótica Diversificada}{Universidade de Aveiro}{}{\vspace{3pt}Fundou e desenvolveu, juntamente com alguns colegas, o NeRD (Núcleo de robótica diversificada}, participando na organização de actividades relacionadas com a robótica, desde a concepção de um robot até à programação de microcontroladores.}
\end{itemize}

\section{Competências}

\vspace{6pt}

\begin{itemize}

\item \textbf{Programming Languages:} Java, C, C++, Python, SQL, Bash, TeX \\ Também já utilizou Assembly, VHDL, Matlab/Octave, C\#.

\vspace{6pt}

\item \textbf{Tecnologias utilizadas:} Qt, OpenGL, Android, Git, Subversion, Java Swing.

\vspace{6pt}

\item \textbf{Outros: } Elevado nível de fluência na língua inglesa (o \textit{paper} escrito para a \textit{ConfTele} foi escrito em inglês; a dissertação também está a ser escrita nesta língua).

\end{itemize}

\section{Interesses e actividades extra-curriculares}

\vspace{6pt}

\begin{itemize}

\item{Possui paixão pelo mundo do sistema operativo GNU/Linux, e do software livre em geral. Planeia ser um membro activo no desenvolvimento e manutenção de projectos de software livre assim que acabar a dissertação.}

\vspace{6pt}

\item{Nos tempos livres, para além de estudar sobre a área, preocupa-se em alargar horizontes, estudando línguas (Alemão) e filosofia. Frequenta assiduamente o ginásio.}

\end{itemize}


% Publications from a BibTeX file without multibib
%  for numerical labels: \renewcommand{\bibliographyitemlabel}{\@biblabel{\arabic{enumiv}}}% CONSIDER MERGING WITH PREAMBLE PART
%  to redefine the heading string ("Publications"): \renewcommand{\refname}{Articles}
\nocite{*}
\bibliographystyle{plain}
\bibliography{publications}                        % 'publications' is the name of a BibTeX file

% Publications from a BibTeX file using the multibib package
%\section{Publications}
%\nocitebook{book1,book2}
%\bibliographystylebook{plain}
%\bibliographybook{publications}                   % 'publications' is the name of a BibTeX file
%\nocitemisc{misc1,misc2,misc3}
%\bibliographystylemisc{plain}
%\bibliographymisc{publications}                   % 'publications' is the name of a BibTeX file

%-----       letter       ---------------------------------------------------------

\end{document}


%% end of file `template.tex'.
