%% start of file `template.tex'.
%% Copyright 2006-2013 Xavier Danaux (xdanaux@gmail.com).
%
% This work may be distributed and/or modified under the
% conditions of the LaTeX Project Public License version 1.3c,
% available at http://www.latex-project.org/lppl/.


\documentclass[11pt,a4paper,sans]{moderncv} % possible options include font size ('10pt', '11pt' and '12pt'), paper size ('a4paper', 'letterpaper', 'a5paper', 'legalpaper', 'executivepaper' and 'landscape') and font family ('sans' and 'roman')

% modern themes
\moderncvstyle{banking}                            % style options are 'casual' (default), 'classic', 'oldstyle' and 'banking'
\moderncvcolor{blue}                                % color options 'blue' (default), 'orange', 'green', 'red', 'purple', 'grey' and 'black'
%\renewcommand{\familydefault}{\sfdefault}         % to set the default font; use '\sfdefault' for the default sans serif font, '\rmdefault' for the default roman one, or any tex font name
%\nopagenumbers{}                                  % uncomment to suppress automatic page numbering for CVs longer than one page

% character encoding
\usepackage[utf8]{inputenc}                       % if you are not using xelatex ou lualatex, replace by the encoding you are using
%\usepackage{CJKutf8}                              % if you need to use CJK to typeset your resume in Chinese, Japanese or Korean

% adjust the page margins
\usepackage[scale=0.75]{geometry}
%\setlength{\hintscolumnwidth}{3cm}                % if you want to change the width of the column with the dates
%\setlength{\makecvtitlenamewidth}{10cm}           % for the 'classic' style, if you want to force the width allocated to your name and avoid line breaks. be careful though, the length is normally calculated to avoid any overlap with your personal info; use this at your own typographical risks...

\usepackage{import}

% personal data
\name{João}{Grego}
\title{Curriculum Vitae}                               % optional, remove / comment the line if not wanted
\phone[mobile]{+351 915569263}                   % optional, remove / comment the line if not wanted
\email{jpedro@grego.pt}                               % optional, remove / comment the line if not wanted
\extrainfo{https://github.com/jpgrego}                         % optional, remove / comment the line if not wanted
\homepage{https://gitlab.com/jpgrego}
%\extrainfo{additional information}                 % optional, remove / comment the line if not wanted
%\photo[64pt][0.4pt]{picture}                       % optional, remove / comment the line if not wanted; '64pt' is the height the picture must be resized to, 0.4pt is the thickness of the frame around it (put it to 0pt for no frame) and 'picture' is the name of the picture file
%\quote{Some quote}                                 % optional, remove / comment the line if not wanted

% to show numerical labels in the bibliography (default is to show no labels); only useful if you make citations in your resume
%\makeatletter
%\renewcommand*{\bibliographyitemlabel}{\@biblabel{\arabic{enumiv}}}
%\makeatother
%\renewcommand*{\bibliographyitemlabel}{[\arabic{enumiv}]}% CONSIDER REPLACING THE ABOVE BY THIS

% bibliography with mutiple entries
%\usepackage{multibib}
%\newcites{book,misc}{{Books},{Others}}
%----------------------------------------------------------------------------------
%            content
%----------------------------------------------------------------------------------
\begin{document}
%\begin{CJK*}{UTF8}{gbsn}                          % to typeset your resume in Chinese using CJK
%-----       resume       ---------------------------------------------------------
\makecvtitle

\small{Working student, currently writing the Master's dissertation to
  complete the course of Computers and Telematics Engineering in the University
  of Aveiro. Passionate for software development and systems administration,
  with strong technical and social skills. Determined to learn with those who
  know more, and to teach those who know less, with sharing information and the
  spreading of knowledge.}

\section{Professional Experience}

\vspace{6pt}

\begin{itemize}

\item{\cventry{July 2018 -- Present}{Systems Engineer}{University of
      Aveiro}{Aveiro}{Information Systems and Communications Group}{Part of the
      team that is responsible for systems management and
      administration. Specializes in the maintenance and troubleshooting of
      machines based on GNU/Linux, as well as the development of new solutions
      that are built upon this operating system. In collaboration with his team,
      he installed a new Glassfish solution, manages the MySQL / MariaDB and
      PostgreSQL databases, and administrates the load balancing solution based
      on Linux Virtual Server. He's also currently one of the members
      responsible for the current VoIP solution, and currently assisting in the
      testing and installation of the new one. Configures web servers like
      Apache across several machines, as well as application servers such as
      Tomcat and the aforementioned Glassfish. One of the main tasks at hand at
      this moment is introducing containerization, using Docker, collaborating
      in the paradigm change from a structure strongly built on virtual
      machines. Also contributing to the instalation on premises of the
      distributed streaming platform Kafka. Lastly, he's part of the ``Computer
      Security Incident Response Team'' (\url{https://csirt.ua.pt/page/24264},
      ``Membros da Equipa'') of the institution, aiding in the analysis of
      cybersecurity incidents and the implementation of security policies.}}

\vspace{6pt}

\item{\cventry{Novembro 2016 -- Julho 2018}{Software Engineer}{University of
      Aveiro}{Aveiro}{Open Platforms Group}{Took part in a innovative team
      resposible for the customization, instalation and maintenance of free and
      open source software solutions to be used by several departments of the
      institution, such as the internal management of information systems
      (\textit{WSO2}) and services available to the public (\textit{Moodle},
      \textit{DSpace}, \textit{OJS}, ...). Produced several bash scripts, some
      of which can be seen on the University of Aveiro GitHub's page
      (\url{https://www.github.com/universidadeaveiro/}). Managed several
      GNU/Linux development machines that belonged to the group, or that the
      group had access to. Before joining this team, he was responsible for the
      development of the backoffice for the mobile application relative to the
      project ``IES + Perto'', improving the already existing web services, and
      creating new ones. He was also responsible for the improvement of
      functionalities and bug fixing of an Android application, to be used by
      the security personnel of the university campus. Consistently elaborated
      documentation using LaTeX and LibreOffice, describing step-by-step the
      installation of platforms and development environments.}}

\vspace{6pt}

\item{\cventry{Abril 2015 -- Outubro 2016}{Software
      Engineer}{BEEVERYCREATIVE}{Aveiro}{}{Joined the team responsible for the
      development of software and firmware used in BEEVERYCREATIVE's printers, a
      portuguese 3D printer designer. Maintained existing projects,
      significantly improving their stability and usability, and developed new
      ones, with collaboration with external entities, such as the European
      Space Agency.}}

\vspace{6pt}

\newpage
\item{\cventry{Julho 2011}{Network technician}{Meo XL
      Party}{Braga}{}{\vspace{3pt}Assisted in the setup and maintenance of the
      Meo XL Party internal network, guaranteeing quality of service and
      communications during the event's activities. Meo XL Party is one of the
      major eSports events in Portugal.}}
  
\end{itemize}

\section{Education}

\vspace{5pt}

\subsection{Academic qualifications}

\vspace{5pt}

\begin{itemize}

\item{\cventry{}{Licenciate degree in Computers and Telematics
      Engineering}{University of Aveiro}{Aveiro}{}{}}

\end{itemize}

\vspace{2pt}

\subsection{Projetos}

\vspace{5pt}

\begin{itemize}

\item{\textbf{Dissertação de mestrado (em progresso): } \textit{'Integrated
      monitoring in Android Devices'}

\vspace{3pt}

\small{Currently elaborating the finishing touches in the elaboration of a
  monitoring system for the Android operating system. It consists in an
  applicaiton with the purpose of analysing the device's activity and,
  afterwards, attempt to detect events that deviate too far from that which is
  established to be the normal operation and behaviour of the cell phone. This
  work is, then, an attempt to present an alternative in the detection of
  malware in mobile devices. Wrote a paper about this work for the
  \textit{ConfTele} conference, organized by the Institute of
  Telecommunications, located in the University of Aveiro campus.}

\item{\textbf{Projeto de 3º ano: } \textit{'Attendance control'}

\vspace{3pt}

\small{Este projeto tinha como objetivo desenvolver um método eficiente para
  registar automaticamente a presença de estudantes em aulas, exames e outras
  atividades académicas relacionadas. No final, o projeto foi apresentado e o
  seu grupo reuniu-se com os Serviços de Tecnologias de Informação e Comunicação
  da Universidade de Aveiro, de maneira a implementar oficialmente o sistema.}}

\vspace{6pt}

\item{\textbf{Projeto da cadeira de Engenharia de Software: }\textit{'Cambada
      in the Web'}}

\vspace{3pt}

\small{Integrou uma equipa no âmbito da cadeira de Engenharia de
  \textit{Software}, em colaboração com o CAMBADA (\textit{Cooperative
    Autonomous Mobile roBots with Advanced Distributed Architecture}), a famosa
  equipa de futebol robótico da Universidade de Aveiro, de maneira a desenvolver
  uma plataforma \textit{web} para observação de estatísticas e visualização e
  gravação de jogos a decorrer.}}

\vspace{6pt}

\item{\textbf{Projeto da cadeira de Engenharia de Serviços: }\textit{'eLock'}}

\vspace{3pt}

\small{O \textit{eLock} foi desenvolvido durante a frequência da cadeira de
  Engenharia de Serviços. Desenvolveu um sistema de fechadura electrónica
  seguro, eficiente e customizável que permitisse a um utilizador abrir a
  fechadura com o seu dispositivo móvel, autorizasse outros utilizadores a
  abrirem a fechadura e que especificasse o horário a que um utilizador poderia
  abrir determinada fechadura.}

\end{itemize}

\section{Outras experiências}

\vspace{6pt}

\begin{itemize}
\item{\cventry{Maio 2012}{Participante}{Micro-rato}{Universidade de
      Aveiro}{}{\vspace{3pt}Participou nesta competição, que consiste em
      programar um \textit{robot} para encontrar o caminho que deve percorrer
      para chegar ao final de um labirinto.}}

\vspace{6pt}

\item{\cventry{2013 -- 2014}{Membro fundador}{Núcleo de Robótica
      Diversificada}{Universidade de Aveiro}{}{\vspace{3pt}Fundou e desenvolveu
      o NeRD (Núcleo de robótica diversificada), participando na organização de
      atividades relacionadas com a robótica, desde a concepção de um
      \textit{robot} até à programação de microcontroladores.}}
\end{itemize}

\section{Competências}

\vspace{6pt}

\begin{itemize}

\item \textbf{Linguagens de programação/\textit{scripting}:} Java, C, C++, Python, Bash, Fish, LaTeX. \\

\vspace{6pt}

\item \textbf{Tecnologias utilizadas:} Ansible, Docker, Git, Subversion.
  
\vspace{6pt}

\item \textbf{Sistemas Operativos:} Gentoo, Debian, Ubuntu, CentOS.

\vspace{6pt}

\item \textbf{Aplicações de servidor:} Apache, nginx, Traefik, Tomcat,
  Glassfish.

\vspace{6pt}

\item \textbf{Bases de Dados:} MySQL / MariaDB, Galera, PostgreSQL.

\vspace{6pt}

\item \textbf{Balanceamento:} Linux Virtual Server (LVS)

\vspace{6pt}

\item \textbf{VoIP:} Asterisk, Kamailio, Acme Packet Net-Net OS (SBC)

\vspace{6pt}

\item \textbf{Outros: } Conhecimento e aptidão para expressões regulares,
  utilizadas principalmente em scripts bash. Elevado nível de fluência na língua
  inglesa (o \textit{paper} escrito para a \textit{ConfTele} foi escrito em
  inglês; a dissertação também está a ser escrita nesta língua).

\end{itemize}

\section{Interesses e actividades extra-curriculares}

\vspace{6pt}

\begin{itemize}

\item{Possui um grande interesse pelo sistema operativo GNU/Linux, utlizando a
    distribuição \textit{Gentoo} no seu computador pessoal e \textit{Debian} nos
    seus computadores de trabalho.  Consequentemente, tem também grande
    afinidade pelo \textit{software} livre em geral e planeia ser um membro
    ativo no desenvolvimento e manutenção de projectos de \textit{software}
    livre.}

\vspace{6pt}

\item{Nos tempos livres, para além de programar, montar serviços nos seus
    servidores pessoais e acompanhar os últimos desenvolvimentos, preocupa-se em
    alargar horizontes, estudando um pouco de filosofia e política.}

\end{itemize}


% Publications from a BibTeX file without multibib
%  for numerical labels: \renewcommand{\bibliographyitemlabel}{\@biblabel{\arabic{enumiv}}}% CONSIDER MERGING WITH PREAMBLE PART
%  to redefine the heading string ("Publications"): \renewcommand{\refname}{Articles}
\nocite{*}
\bibliographystyle{plain}
\bibliography{publications}                        % 'publications' is the name of a BibTeX file

% Publications from a BibTeX file using the multibib package
%\section{Publications}
%\nocitebook{book1,book2}
%\bibliographystylebook{plain}
%\bibliographybook{publications}                   % 'publications' is the name of a BibTeX file
%\nocitemisc{misc1,misc2,misc3}
%\bibliographystylemisc{plain}
%\bibliographymisc{publications}                   % 'publications' is the name of a BibTeX file

%-----       letter       ---------------------------------------------------------

\end{document}


%% end of file `template.tex'.

%%% Local Variables:
%%% mode: latex
%%% TeX-master: t
%%% End:
