%% start of file `template.tex'.
%% Copyright 2006-2013 Xavier Danaux (xdanaux@gmail.com).
%
% This work may be distributed and/or modified under the
% conditions of the LaTeX Project Public License version 1.3c,
% available at http://www.latex-project.org/lppl/.


\documentclass[11pt,a4paper,sans]{moderncv} % possible options include font size ('10pt', '11pt' and '12pt'), paper size ('a4paper', 'letterpaper', 'a5paper', 'legalpaper', 'executivepaper' and 'landscape') and font family ('sans' and 'roman')

% modern themes
\moderncvstyle{banking}                            % style options are 'casual' (default), 'classic', 'oldstyle' and 'banking'
\moderncvcolor{blue}                                % color options 'blue' (default), 'orange', 'green', 'red', 'purple', 'grey' and 'black'
%\renewcommand{\familydefault}{\sfdefault}         % to set the default font; use '\sfdefault' for the default sans serif font, '\rmdefault' for the default roman one, or any tex font name
%\nopagenumbers{}                                  % uncomment to suppress automatic page numbering for CVs longer than one page

% character encoding
\usepackage[utf8]{inputenc}                       % if you are not using xelatex ou lualatex, replace by the encoding you are using
%\usepackage{CJKutf8}                              % if you need to use CJK to typeset your resume in Chinese, Japanese or Korean

% adjust the page margins
\usepackage[scale=0.75]{geometry}
%\setlength{\hintscolumnwidth}{3cm}                % if you want to change the width of the column with the dates
%\setlength{\makecvtitlenamewidth}{10cm}           % for the 'classic' style, if you want to force the width allocated to your name and avoid line breaks. be careful though, the length is normally calculated to avoid any overlap with your personal info; use this at your own typographical risks...

\usepackage{import}

% personal data
\name{João}{Grego}
\title{Curriculum Vitae}                               % optional, remove / comment the line if not wanted
\phone[mobile]{+351 915569263}                   % optional, remove / comment the line if not wanted
\email{jpedro@grego.pt}                               % optional, remove / comment the line if not wanted
\extrainfo{https://github.com/jpgrego}                         % optional, remove / comment the line if not wanted
\homepage{https://gitlab.com/jpgrego}
%\extrainfo{additional information}                 % optional, remove / comment the line if not wanted
%\photo[64pt][0.4pt]{picture}                       % optional, remove / comment the line if not wanted; '64pt' is the height the picture must be resized to, 0.4pt is the thickness of the frame around it (put it to 0pt for no frame) and 'picture' is the name of the picture file
%\quote{Some quote}                                 % optional, remove / comment the line if not wanted

% to show numerical labels in the bibliography (default is to show no labels); only useful if you make citations in your resume
%\makeatletter
%\renewcommand*{\bibliographyitemlabel}{\@biblabel{\arabic{enumiv}}}
%\makeatother
%\renewcommand*{\bibliographyitemlabel}{[\arabic{enumiv}]}% CONSIDER REPLACING THE ABOVE BY THIS

% bibliography with mutiple entries
%\usepackage{multibib}
%\newcites{book,misc}{{Books},{Others}}
%----------------------------------------------------------------------------------
%            content
%----------------------------------------------------------------------------------
\begin{document}
%\begin{CJK*}{UTF8}{gbsn}                          % to typeset your resume in Chinese using CJK
%-----       resume       ---------------------------------------------------------
\makecvtitle

\small{Trabalhador-estudante e aluno finalista do mestrado integrado de
  Engenharia de Computadores e Telemática, presentemente a desenvolver a
  dissertação de mestrado. Nutre paixão por desenvolvimento de software e
  gestão de infraestrutura informática e redes, com fortes competências técnicas
  e sociais. Tem gosto em aprender com quem sabe mais, ensinar quem sabe menos,
  em partilhar informação e disponibilizar conhecimento.}

\section{Experiência profissional}

\vspace{6pt}

\begin{itemize}

\item{\cventry{Julho 2018 -- Presente}{Engenheiro de Sistemas}{Universidade de
      Aveiro}{Aveiro}{Área de Segurança, Informática e Comunicações}{Integra a
      equipa responsável por gestão de servidores. É responsável pela manutenção
      e \textit{troubleshooting} de máquinas com base em GNU/Linux bem como o
      desenvolvimento de novas soluções assentes no mesmo sistema operativo. Em
      colaboração com os membros da equipa, instalou uma nova solução de
      \textit{Glassfish}, faz a gestão de bases de dados \textit{MySQL /
        MariaDB} e \textit{PostgreSQL}, gere a solução de \textit{VoIP} e
      administra a solução de balanceamento \textit{Linux Virtual
        Server}. Configura o servidor \textit{Apache} em diversas máquinas, bem
      como servidores de aplicações \textit{Tomcat} e o supramencionado
      \textit{Glassfish}. Encontra-se também a introduzir
      \textit{containerização} usando \textit{Docker}, colaborando na mudança de
      paradigma de uma estrutura baseada em máquinas virtuais na infraestrutura
      da Universidade de Aveiro. Integra a \textit{Computer Security Incident
        Response Team} (CSIRT) da instituição, colaborando na análise de
      incidentes de cibersegurança e implementação de políticas.}}

\vspace{6pt}

\item{\cventry{Novembro 2016 -- Julho 2018}{Engenheiro de Software}{Universidade
      de Aveiro}{Aveiro}{Área de Plataformas Abertas}{Fez parte de uma equipa
      responsável pelo desenvolvimento, instalação e manutenção de plataformas
      abertas a serem usadas por áreas diversas, como a gestão interna dos
      serviços de informática e serviços disponíveis ao público
      (\textit{Moodle}, \textit{DSpace}, \textit{OJS}, ...). Produziu diversos
      scripts em bash, alguns dos quais podem ser consultados na página do
      GitHub da Universidade de Aveiro
      (\href{https://www.github.com/universidadeaveiro/moodle-moosh-scripts}
      {universidadeaveiro/moodle-mosh-scripts}). Para além disso, geriu diversas
      máquinas GNU/Linux de testes pertencente ao grupo, ou a que o grupo teve
      acesso. Antes de ingressar nesta equipa, esteve responsável pelo
      desenvolvimento do \textit{backoffice} da aplicação móvel relativa ao
      projecto ``IES + Perto'', melhorando serviços já existentes e criando
      novos. Esteve também responsável pela melhoria de funcionalidades e
      correcção de \textit{bugs} numa aplicação para o sistema operativo móvel
      \textit{Android} para os vigilantes do \textit{campus} universitário. Esta
      última atribuía rondas a serem desempenhadas, \textit{check-in} de
      pontos-chaves, envio de fotografias de ocorrências a registar e emissão de
      alertas. Elaborou frequentemente documentação em LaTeX e LibreOffice,
      descrevendo os passos para instalar certas plataformas e instalação de
      ambientes de desenvolvimento.}}

\vspace{6pt}

\item{\cventry{Abril 2015 -- Outubro 2016}{Engenheiro de
      Software}{BEEVERYCREATIVE}{Aveiro}{}{Integrou a equipa responsável pelo
      desenvolvimento de \textit{software} e \textit{firmware} das impressoras
    da BEEVERYCREATIVE, empresa fabricante de impressoras 3D. Efectuou a
    manutenção de projectos existentes, melhorando significativamente a sua
    estabilidade e usabilidade, e desenvolveu projectos novos, em colaboração
    com entidades externas, como a Agência Espacial Europeia.}}

\vspace{6pt}

\item{\cventry{Julho 2011}{Técnico de Redes}{Meo XL
      Party}{Braga}{}{\vspace{3pt}Integrou uma equipa responsável pela montagem
      e manutenção da rede interna do evento, tendo que garantir serviço e
      qualidade do sinal durante o seu decorrer.}}
  
\end{itemize}

\section{Educação}

\vspace{5pt}

\subsection{Qualificações académicas}

\vspace{5pt}

\begin{itemize}

\item{\cventry{}{Licenciatura em Engenharia de Computadores e
      Telemática}{Universidade de Aveiro}{Aveiro}{}{}}

\end{itemize}

\vspace{2pt}

\subsection{Projetos}

\vspace{5pt}

\begin{itemize}

\item{\textbf{Dissertação de mestrado (em progresso): } \textit{'Integrated
      monitoring in Android Devices'}

\vspace{3pt}

\small{Encontra-se de momento na fase de conceção e elaboração de um sistema de
  monitorização para o sistema operativo de dispositivos móveis
  \textit{Android}. Este sistema consiste numa aplicação que tem como objetivo
  analisar a atividade do dispositivo em que se encontra instalado para,
  posteriormente, tentar detetar eventos que se afastem demasiado daquilo que é
  estabelecido ser o comportamento normal do dispositivo, o seu \textit{status
    quo} de operação. Este trabalho é, assim, uma tentativa de apresentar uma
  alternativa na deteção de \textit{malware} em dispositivos móveis. Escreveu um
  \textit{paper} sobre este trabalho para a \textit{ConfTele}, conferência que
  decorreu em setembro de 2015, organizado pelo Instituto de Telecomunicações,
  na Universidade de Aveiro.}

\item{\textbf{Projeto de 3º ano: } \textit{'Attendance control'}

\vspace{3pt}

\small{Este projeto tinha como objetivo desenvolver um método eficiente para
  registar automaticamente a presença de estudantes em aulas, exames e outras
  atividades académicas relacionadas. No final, o projeto foi apresentado e o
  seu grupo reuniu-se com os Serviços de Tecnologias de Informação e Comunicação
  da Universidade de Aveiro, de maneira a implementar oficialmente o sistema.}}

\vspace{6pt}

\item{\textbf{Projeto da cadeira de Engenharia de Software: }\textit{'Cambada
      in the Web'}}

\vspace{3pt}

\small{Integrou uma equipa no âmbito da cadeira de Engenharia de
  \textit{Software}, em colaboração com o CAMBADA (\textit{Cooperative
    Autonomous Mobile roBots with Advanced Distributed Architecture}), a famosa
  equipa de futebol robótico da Universidade de Aveiro, de maneira a desenvolver
  uma plataforma \textit{web} para observação de estatísticas e visualização e
  gravação de jogos a decorrer.}}

\vspace{6pt}

\item{\textbf{Projeto da cadeira de Engenharia de Serviços: }\textit{'eLock'}}

\vspace{3pt}

\small{O \textit{eLock} foi desenvolvido durante a frequência da cadeira de
  Engenharia de Serviços. Desenvolveu um sistema de fechadura electrónica
  seguro, eficiente e customizável que permitisse a um utilizador abrir a
  fechadura com o seu dispositivo móvel, autorizasse outros utilizadores a
  abrirem a fechadura e que especificasse o horário a que um utilizador poderia
  abrir determinada fechadura.}

\end{itemize}

\section{Outras experiências}

\vspace{6pt}

\begin{itemize}
\item{\cventry{Maio 2012}{Participante}{Micro-rato}{Universidade de
      Aveiro}{}{\vspace{3pt}Participou nesta competição, que consiste em
      programar um \textit{robot} para encontrar o caminho que deve percorrer
      para chegar ao final de um labirinto.}}

\vspace{6pt}

\item{\cventry{2013 -- 2014}{Membro fundador}{Núcleo de Robótica
      Diversificada}{Universidade de Aveiro}{}{\vspace{3pt}Fundou e desenvolveu
      o NeRD (Núcleo de robótica diversificada), participando na organização de
      atividades relacionadas com a robótica, desde a concepção de um
      \textit{robot} até à programação de microcontroladores.}}
\end{itemize}

\section{Competências}

\vspace{6pt}

\begin{itemize}

\item \textbf{Linguagens de programação/\textit{scripting}:} Java, C, C++, Python, Bash, Fish, LaTeX. \\

\vspace{6pt}

\item \textbf{Tecnologias utilizadas:} Ansible, Docker, Git, Subversion.
  
\vspace{6pt}

\item \textbf{Sistemas Operativos:} Gentoo, Debian, Ubuntu, CentOS.

\vspace{6pt}

\item \textbf{Aplicações de servidor:} Apache, nginx, Traefik, Tomcat,
  Glassfish.

\vspace{6pt}

\item \textbf{Bases de Dados:} MySQL / MariaDB, Galera, PostgreSQL.

\vspace{6pt}

\item \textbf{Balanceamento:} Linux Virtual Server (LVS)

\vspace{6pt}

\item \textbf{VoIP:} Asterisk, Kamailio, Acme Packet Net-Net OS (SBC)

\vspace{6pt}

\item \textbf{Outros: } Conhecimento e aptidão para expressões regulares,
  utilizadas principalmente em scripts bash. Elevado nível de fluência na língua
  inglesa (o \textit{paper} escrito para a \textit{ConfTele} foi escrito em
  inglês; a dissertação também está a ser escrita nesta língua).

\end{itemize}

\section{Interesses e actividades extra-curriculares}

\vspace{6pt}

\begin{itemize}

\item{Possui um grande interesse pelo sistema operativo GNU/Linux, utlizando a
    distribuição \textit{Gentoo} no seu computador pessoal e \textit{Debian} nos
    seus computadores de trabalho.  Consequentemente, tem também grande
    afinidade pelo \textit{software} livre em geral e planeia ser um membro
    ativo no desenvolvimento e manutenção de projectos de \textit{software}
    livre.}

\vspace{6pt}

\item{Nos tempos livres, para além de programar, montar serviços nos seus
    servidores pessoais e acompanhar os últimos desenvolvimentos, preocupa-se em
    alargar horizontes, estudando um pouco de filosofia e política.}

\end{itemize}


% Publications from a BibTeX file without multibib
%  for numerical labels: \renewcommand{\bibliographyitemlabel}{\@biblabel{\arabic{enumiv}}}% CONSIDER MERGING WITH PREAMBLE PART
%  to redefine the heading string ("Publications"): \renewcommand{\refname}{Articles}
\nocite{*}
\bibliographystyle{plain}
\bibliography{publications}                        % 'publications' is the name of a BibTeX file

% Publications from a BibTeX file using the multibib package
%\section{Publications}
%\nocitebook{book1,book2}
%\bibliographystylebook{plain}
%\bibliographybook{publications}                   % 'publications' is the name of a BibTeX file
%\nocitemisc{misc1,misc2,misc3}
%\bibliographystylemisc{plain}
%\bibliographymisc{publications}                   % 'publications' is the name of a BibTeX file

%-----       letter       ---------------------------------------------------------

\end{document}


%% end of file `template.tex'.

%%% Local Variables:
%%% mode: latex
%%% TeX-master: t
%%% End:
